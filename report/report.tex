\documentclass[journal]{IEEEtran}

\hyphenation{op-tical net-works semi-conduc-tor}

\usepackage{cite}
\usepackage{amsmath,amssymb,amsfonts}
\usepackage{algorithmic}
\usepackage{graphicx}
\usepackage{textcomp}
\usepackage{xcolor}
\usepackage[linesnumbered,ruled,vlined]{algorithm2e}
\usepackage{hyperref}

\begin{document}

\title{RAID-6 Based Distributed Storage System}

\author{Meng Shen,
        Ruihang Wang,
        Yihang Li % <-this % stops a space
\thanks{Meng shen and Ruihang Wang are with the School
of Computer Science Engineering, Nanyang Technological University, Singapore,
e-mail: (meng005@e.ntu.edu.sg, ruihang001@e.ntu.edu.sg).}% <-this % stops a space
\thanks{Yihang Li is with the School
of Electrical and Electronic Engineering, Nanyang Technological University, Singapore,
e-mail: (leah9704@gmail.com).}% <-this % stops a space
\thanks{Project submitted on Nov 23, 2019.}}

% The paper headers
\markboth{CE7490 2019 Fall - Advanced Topics in Distributed System - Project 2}%
{Shell \MakeLowercase{\textit{et al.}}: Bare Demo of IEEEtran.cls for IEEE Journals}

% make the title area
\maketitle

\begin{abstract}
RAID-6 is an extension of RAID-5 by adding another parity block. It is designed to tolerate any two concurrent disk failure and maintain a relatively high storage efficiency. The mathematics behind RAID-6 is Galois Field theory and Reed-Solomon Coding. In this project, we create our own finite field operation library and use it to develop a RAID-6 based distributed storage system in Python 3.7 under Linux envrionment. In addition to basic functionalities, we also developed several advanced features in the system that can accommodate real files of arbitrary size and support user defined configurations. To make it available for implementation, we open-source our codes with read-me documents at \url{https://github.com/GuluDeemo/CE7490-RAID6}
\end{abstract}

% Note that keywords are not normally used for peerreview papers.
\begin{IEEEkeywords}
RAID-6, Galois Field, Reed-Solomon codes, Python. 
\end{IEEEkeywords}


\section{Introduction}
\IEEEPARstart{T}{he} concept of RAID (Redundant Array of Independent Disks) was raised in the paper "A Case for Redundant Arrays of Inexpensive Disks (RAID)" for the demand for rising the reliability, capacity and speed of storage systems. Instead of reading and writing data in one disk drive, RAID achieves larger width of input/output, and hence improves the performance significantly. Many levels of RAID have been raised since 1970s and are widely used in different data protection requirements. Most RAID implement error-detection drives and employ error protection schemes called "parity" to provide fault tolerance in one or more disks. Most use simple XOR, but RAID-6 uses two separate parities based respectively on addition and multiplication in a Galois Field and Reed-Solomon error detection. This report focused on RAID-6.

Among all levels of RAID, RAID-5 and RAID-10 are most widely used by industry manufacturers. RAID-10 combines RAID-1 and RAID-0 by dividing data into several parts and mirroring each part into pairs. RAID-10 is designed to tolerate multiple disk failure except two corresponding disks in the mirroring pairs. However, the storage efficiency is at only 50\%, which is very low. RAID-5 consists of block-level striping with distributed parity, and the parity information is saved among drives. When facing failure, the loss data can be calculated and rebuilt by distributed parity. However, RAID-5 can only tolerate one disk failure. RAID-6 is designed to tolerate any two concurrent disk failure and maintain a relatively high storage efficiency.

RAID-6 is a private raid-level standard proposed concurrently by several large enterprises. This RAID level is developed on the basis of RAID-5, while unlike RAID-5, there is not only a parity area for sibling data on the drive, but also a parity area for each data block. Two independent parity systems use different algorithms and the data is very reliable. Even if two disks fail at the same time, it does not affect the use of the data, which enhances the disk’s fault tolerance. With a RAID-6 based distributed storage system, it is possible to mitigate most of the problems associated with RAID-5.

RAID-6 achieves faster reading performance and higher fault tolerance. All these advantages make larger RAID groups more practical, especially for high-availability systems, as larger-capacity drives take longer to restore. The larger the drive capacities and the larger the array size, the more important it becomes to choose RAID-6 instead of RAID-5. However, RAID-6 is not as popular as other levels of RAID for practical application because of its complicated system and expensive RAID controller. For most small businesses which do not require high security level of data preservation, it is better to use RAID-5 economically. While for businesses demand higher security level of data preservation like data center, it is necessary to implement RAID-6.

In this report, we present a reliable distributed storage system based on RAID-6 under Python 3.7 environment. The mathematical operation in Galois Field is well investigated and created as a library to support RAID-6 controller through practical implementation. The RAID-6 based system support several basic functionalities, such as distributed data storage, failure detection and lost redundancy recovery. On top of the minimal implementation, we also investigate ...
The rest of the report is organized as follows: ...

\section{Overview}
The RAID-6 distributed storage system is developed in Python 3.7 under Linux environment. Instead of using existing finite field libraries, we create our own Galois Field \textbf{GF($2^8$)} operations based on the tutorial document of Reed-Solomon Coding in \cite{bibid}. The only dependency we use in this project is \emph{NumPy} for basic matrix operations. The functionalities we have developed for this RAID-6 system are summarized as follows:
\begin{itemize}
	\item Store and access abstract “data objects” across storage nodes using RAID-6 for fault-tolerance.
	\item Include mechanisms to determine failure of storage nodes.
	\item Carry out rebuild of lost redundancy at a replacement storage node.
	\item Accommodate real files of arbitrary size, taking into account issues like RAID mapping, etc. 
	\item Support mutable files, taking into account update of the content, and consistency issues.
	\item Support larger set of configurations
\end{itemize}

\section{Problem Specification}
The RAID-6 theory is based on the Galois Field mathematics and the Reed-Solomon codes. This section will provide relevant definations and principles in the implementation of a RAID-6 distributed storage system. First, let $D_1, D_2, ... , D_n$ denote $n$ \emph{storage disks} and $C_1, C_2, ... , C_m$ denote $m$ \emph{checksum disks}. Each of this disk holds the same storage capacity. The objective is to define the calculation of $C_i$ such that if any $m$ of $D_1, D_2, ... , D_n, C_1, C_2, ... , C_m$ corrupt. The contents of the corrupted devices can be rebuilt from the non-corrupted devices.

\subsection{Arithmetic over Galois Field}
Fields with $2^w$ elements under closed operations like addition and multiplication are called \emph{Galois Fields} (denoted as \textbf{GF($2^w$)}). The elements are integers from zero to $2^w-1$. For example, the field \textbf{GF(2)} can be represent as the set ${0,1}$. A small field would limit the number of disks possible while a large field would require extremely large tables. For RAID-6, we choose the commonly used field of \textbf{GF($2^8$)}, which allows for a maximum of 255 data disks. The operations in this finite field are illustrated as below:

\emph{Addition}: The addition field operator $(\oplus)$ is  performed by bitwise XOR. 

\emph{Subtraction}: The addition and subtraction $(\ominus)$ are the same operation. For example, $A \oplus B = A \ominus B$

\emph{Multiplication}: The multiplication $(\otimes)$ is performed by bitwise AND, and can be simplified using logarithms table. The details of setting up this table will be presented later.

\emph{Division}: Division is defined as multiplication with an inverse like $A/B = A \otimes B^{-1}$

Based on the definitions of the foundamental operations, the following basic rules should be obeyed in this field:

\begin{itemize}
	\item Addition is commutative: $A \oplus B = B \oplus A$
	\item Addition is associative:    $(A \oplus B) \oplus C = A \oplus (B \oplus C) $
	\item Multiplication is commutative: $A \otimes B = B \otimes A$
	\item Multiplication is associative: $(A \otimes B) \otimes C = A \otimes (B \otimes C)$
	\item  Distributive law: $(A \oplus B) \otimes C = A \otimes C  \oplus B \otimes C$
\end{itemize}

Here, we further explain how to perform multiplication and division in \textbf{GF($2^8$)}. The multiplication and division in this field is much more difficult and harder to implement. Multiplying two field elements is to multiply their corresponding polynominals. To simplify this calculation, we use two look up logarithm tables with length of each equals $2^w-1$. These tables are defined as \emph{gflog} and \emph{gfilog} and can be generated using Algorithm 1.

\begin{itemize}
	\item \emph{gflog[]}: This table is defined for the indices 1 to 255, which maps the index to its logarithm in the Galois Field.
	\item \emph{gfilog[]}: This table is defined for the indices 0 to 254, which maps the index to its inverse logarithm in the Galois Field. 
\end{itemize}

Obviously, \emph{gflog[gfilog[i]] = i}, and \emph{gfilog[gflog[i]] =i}. With these two tables, we can easily multiply two non-zero elements of \textbf{GF($2^8$)} by adding their logs and then taking the inverse log as given by:

\begin{equation}
A \otimes B = gfilog[gflog[A] + gflog[B]]
\end{equation}

\begin{equation}
A \otimes B^{-1} = gfilog[gflog[A] - gflog[B]]
\end{equation}

When $w$ is small, these two tables can accelerate multiplication and division in the Galois Field.

\begin{algorithm}
	\KwIn{$w=8$, $modulus=0b100011101$}
	\KwOut{two logarithm tables}
	max = $1<<8$ \\
	$b = 1$ \\
	\emph{gflog}, \emph{gfilog} = malloc(max)\\
	\For{$log = 0$ \KwTo max}
	{
		\emph{gflog}[$b$] = $log$\\
		\emph{gfillog}[$log$] = $b$\\
		$b$ = $b<<1$\\
		\If{$b$ \& max}{$b$ = $b$ \textsuperscript{$ \wedge $} $modulus$}
	}
	\caption{Setup logarithm tables}
\end{algorithm}

\subsection{Reed-Solomen Coding}

By using Reed-Solomen Coding, we could use more parities to back up our data, not just 2. It could be very helpful when one wants to achieve better data security.

\subsubsection{Calculating Parities}

Suppose we have $N$ data disks and $M$ parity disks. To calculate the parity $P_i$, we define function $F_i$: 

\begin{equation}
P_i = F_i(d_1,d_2,...,d_n) = \sum_{j=1}^{n}d_j f_{i,j}
\end{equation}

If we treat data and parity as vectors, where $D = [d_1, d_2, ... , d_i]$ and $P = [p_1, p_2, ... ,p_i]$, we could get a simpler equation:

\begin{equation}
P=FD
\end{equation}

To calculate P, first, we need to generate the $M \times N$ Vandermonde matrix F where $f_{i,j} = j ^ {i-1}$

\begin{equation}
F =
\left[
\begin{matrix}
1      & 1      & \cdots & 1      \\
1      & 2      & \cdots & N      \\
\vdots & \vdots & \ddots & \vdots \\
1      & 2^{M-1}      & \cdots & N^{M-1}      \\
\end{matrix}
\right]
\end{equation}

In this way, we could easily get our parities $P$ by applying a matrix multiplication on Galois Field.

\subsubsection{Recovering Data}

To recover our data from disk faliure, we need to define matrix $A = \left[\begin{matrix}I \\F \\\end{matrix}\right]$ and $E= \left[\begin{matrix}D \\P \\\end{matrix}\right]$, where $I$ is a $N \times N$ identity matrix and matrix $F, D, P$ are defined before. So we get the equation:

\begin{equation}
AD=E
\end{equation}

When a disk or disks encount failure, we could delete the corresponding row in $A$ and $E$. For instance, we have one failed data disk $D_i$ and one failed parity disk $P_j$, then we need to delete $i_{th}$ row in $I$ and $j_{th}$ row in $F$ to get a new $A'$. Likewise, we also need to delete $i_{th}$ row in $D$ and $j_{th}$ rwow in $P$ to get a new $E'$. Then the equation becomes: 

\begin{equation}
A'D=E'
\end{equation}

What we want is actually D, so the problem becomes to solve the following equation:

\begin{equation}
D=A'^{-1}E'
\end{equation}

The way we use to calculate the inverse matrix $A'^{-1}$ is Gaussian Elimination, the only difference is we do it in Galois Field. Once we calculate our data D, then by using the method mentioned before: $P=FD$, we could update our parity P if it's necessary. 

\section{ Implementation}

\subsection{System Configuration}
The configuration of the RAID-6 based distributed storage system is shown below
\begin{itemize}
	\item Number of Disk: $N$ = 6
	\item Number of Data Disk: $D$ = 4
	\item Number of Checksum Disk: $C$ = 2
	\item $D_1$ ~ $D_4$ denote 4 data disks
	\item $C_1$ ~ $C_4$ denote 2 data disks 
\end{itemize}



\section{Conclusion}
The conclusion goes here.



\appendices
\section{Proof of the First Zonklar Equation}
Appendix one text goes here.


\section{}
Appendix two text goes here.


% use section* for acknowledgment
\section*{Acknowledgment}


The authors would like to thank...

% Can use something like this to put references on a page
% by themselves when using endfloat and the captionsoff option.
\ifCLASSOPTIONcaptionsoff
  \newpage
\fi


\begin{thebibliography}{1}

\bibitem{IEEEhowto:kopka}
H.~Kopka and P.~W. Daly, \emph{A Guide to \LaTeX}, 3rd~ed.\hskip 1em plus
  0.5em minus 0.4em\relax Harlow, England: Addison-Wesley, 1999.

\end{thebibliography}


\end{document}


